%%%%%%%%%%%%%%%%%%%%%%%%
%
% $Autor: Wings $
% $Datum: 2020-07-24 09:05:07Z $
% $Pfad: GDV/Vortraege/latex - Ausarbeitung/Kapitel/Einleitung.tex $
% $Version: 4732 $
%
%%%%%%%%%%%%%%%%%%%%%%%%

\chapter{Positioning Your Analytics}

\section{Existing Solutions}

The analytics developed for Orgadata’s SimplyTag system are positioned uniquely when compared to conventional tools. Traditional solutions like Intrusion Detection Systems (IDS) \cite{IEEE:2020} and log monitoring platforms (e.g., Splunk, ELK Stack) \cite{IEEE:2022} offer general frameworks for detecting anomalies and breaches. However, they often lack customization tailored to specific applications. Orgadata’s approach is distinguished by its precise utilization of trace IDs, HTTP status codes, user agent patterns, and paths to monitor requests in real-time.

\section{State-of-the-Art}

\subsection{Overview of Current Solutions}
\begin{itemize}
	\item Existing tools such as Splunk, ELK Stack and Intrusion Detection Systems (IDS) provide robust frameworks for monitoring and analyzing security events. These solutions excel at processing vast amounts of log data and identifying potential threats in general scenarios.
	\item Security platforms often use rule-based or heuristic approaches for anomaly detection but may struggle with domain-specific customizations.
\end{itemize}

\subsection{Capabilities and Limitations}
\begin{itemize}
	\item \textbf{Capabilities:} Tools like Splunk and ELK Stack provide scalability, integration options, and comprehensive dashboards for security analytics. IDS focuses on real-time threat detection.
	\item \textbf{Limitations:}  Lack of precise tailoring for SimplyTag’s requirements, such as analyzing specific HTTP status codes and user agent patterns. High false-positive rates and inability to integrate seamlessly with Orgadata’s infrastructure are additional challenges.
\end{itemize}

\subsection{Relevance to Analytics}

The SimplyTag analytics focus on filling these gaps by leveraging domain-specific insights, such as monitoring trace IDs and HTTP response patterns to identify anomalies and unauthorized activity.

\begin{enumerate}
	\item The use of trace IDs enables seamless tracking of individual requests across logs, allowing for detailed insights into potential vulnerabilities.
	\item By monitoring HTTP status codes, the system identifies and flags suspicious patterns, such as unexpected 200 codes for unauthorized paths.
	\item Integration of user agent analysis ensures that illegitimate devices or configurations can be quickly identified and addressed.
\end{enumerate}

\subsection{Emerging Trends}
\begin{itemize}
	\item AI-driven anomaly detection is gaining traction, enabling systems to learn and adapt to evolving threats.
	\item Zero-trust architecture and real-time monitoring advancements align with SimplyTag’s goals of enhancing data security and operational reliability.
\end{itemize}

This focused methodology bridges gaps left by traditional systems, providing a solution that is both targeted and scalable for SimplyTag’s operational needs.
